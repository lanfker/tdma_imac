\section{Todo List}
\begin{enumerate}
\item (done) $\implies$ Only relay the ER info item when a node is in the oringal control packet sender's ER.
\item (done) $\implies$ Decide which ER info item to store in the payload of the control packet (item priority).
\item (done) $\implies$ quantile estimation.
\item (done) $\implies$ compute the priority of control packet based on payload item priorities, use the max one.
\item (done) $\implies$ Get rid of meaningful numbers in .cc file, define them in .h files.
\item (done) $\implies$ $d_0$ sampling (first category sampling).
\item (done) $\implies$ 8 levels of priorities in control packets.
\item (done) $\implies$ remove PDR intial values, do EWMA estimation only when a node has the estimated PDR values.
\item (done) $\implies$ Record the reception itme of each ER info item.
\item (done) $\implies$ check which category the info item belongs to.
\item (done) $\implies$ Quantile calculation and EWMA estimation for $d_0$ values of the 2,3 categories.
\item (done) $\implies$ Change 1.5 transmission range as the intial ER edge.
\item (done) $\implies$ when received an ER info item, check which $d0_{quantile}$ to use (by checking categories) and maintain $d_{cat2}$ and $d0_{cat3}$ for each link.
\item (done) $\implies$ piggyback ER info item reception time when transmit control signals.
\item (done) $\implies$ piggyback $d_0$ values for the category 2 and 3.
\item (done) $\implies$ Figured out the memory issue. sampling issue
\item (done) $\implies$ drop ER info items if the current node is not in the ER specified by the ER info item link.
\item (done) $\implies$ report control message reception only if a node is in the control message sender's Max ER.
\item (done) $\implies$ implement the round-robin mechanism when choosing which er info item to transmit
\item (done) $\implies$ change the $d_0$ sampling for the category 2 and 3 ER info items. When receiving a new Er info item of category 2 or 3, do not immediately sample $d_0$ for this item.
\item (done) $\implies$ when power control, add another $\Delta SNR$.
\item (done) $\implies$ For a node, if there are other nodes sending data to it or it will receive data from others, we consider all these nodes, including the node itself, if any of them is in the ER defined by the ER info item, we relay this ER info item;
\item (done) $\implies$ Delay PDR estimation $d_0$ timeslot to enforce the right controller behavior.
\item (done) $\implies$ Use the estimation of N+I, not just the latested received one, or, use pencentile as conservative, or use max. (currently using ewma estimation)

%\item Check the interference sampling and check if the sampled interference values are sent back timely.
%\item (doing) Check the statistics of the interfernece sampling in the control channel to see if it works correctly.
%\item Run the simulation with two versions, one with power control, another with no power control.
\item (done) $\implies$ Compute the data packet loss due to inconsistency.
\item (done) $\implies$ There may be some channel sensing issues, which caused the large $N+I$ samples. Use difference quantile numbers to filter the large $N+I$ samples
\item (done) $\implies$ When do link estimation, use $payloadItem->d\_0$, not use $m\_d0ForCategoryOne$.
\item (done) $\implies$ Do not use $d_0$ ewma estimation, just use the quantile value of $d_0$.
\item (done) $\implies$ When decide if a node should receive an Er infomation itme, first check if this node is in control message sender's ER, then check if this node is in the original control message sender's ER. Finally, check if there is any node with whom this node could send data to or receive data from is in the two Er's.
\item check $d_0$ enforcement. 
\item When do power control, make sure $EVERY$ node can receive the packet.
\item Check $d_0$ quantile estimation accuracy.
\item When doing sliding window $N+I$ estimation, make \emph{two} copy of $N+I$ samples, then sort one according to $NI$ value, sort another one according to sampling time.
\item change initial $ER$ value setting
\item Disable $ACK$ $ER$ since we want to make the concurrency in iMAC close to that in iOrder.
\item Always use the larger $ER$ between the current er and the previous er when forwarding an er information item.  (done)
\item Consider about the lifetime for each er information item.
\item Print out the $d_0$ for the two categories, see the difference. We do not want the different to be too large. 
\end{enumerate}

\section{Different Scenarios}
\begin{enumerate}
  \item use different quantiles of $N+I$ samples, check which one works better for us.
  \item use different quantiles for $d_0$ samples
  \item use power control or not
  \item use $d_0$ or not
  \item use different estimation windows size for $PDR$ estimation.
\end{enumerate}
